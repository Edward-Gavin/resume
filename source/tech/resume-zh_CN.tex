% !TEX TS-program = xelatex
% !TEX encoding = UTF-8 Unicode
% !Mode:: "TeX:UTF-8"

\documentclass{resume}
\usepackage{zh_CN-Adobefonts_external} % Simplified Chinese Support using external fonts (./fonts/zh_CN-Adobe/)
%\usepackage{zh_CN-Adobefonts_internal} % Simplified Chinese Support using system fonts
\usepackage{linespacing_fix} % disable extra space before next section
\usepackage{cite}
\usepackage{fontawesome}

\begin{document}
\pagenumbering{gobble} % suppress displaying page number

\name{王誓伟}

% {E-mail}{mobilephone}{homepage}
% be careful of _ in emaill address
\contactInfo{(+86) 157-3325-3136}{ wsw\_it\_3325@163.com}{Java大数据开发工程师}{BeiJing}
% {E-mail}{mobilephone}
% keep the last empty braces!
%\contactInfo{xxx@yuanbin.me}{(+86) 131-221-87xxx}{}
 
\section{\faTags 个人总结}
擅于发现并解决问题,工作负责,自我驱动力强,好奇心强,热爱尝试新技术。熟悉推荐系统架构、推荐应用技术。具有大规模数据存储挖掘经验,熟悉常用大数据组件。

\section{\faGraduationCap\ 教育背景}
\datedsubsection{\textbf{北京工业大学},信息学部-计算机学院-计算机技术,\textit{硕士 \textbf{ 前5\%}}}{2018.9-2021.6}
\datedsubsection{\textbf{河北农业大学},信息科学与技术学院-网络工程,\textit{学士 \textbf{ 前10\%}}}{2013.9-2017.6}

% \section{\faCogs\ IT 技能}
\section{\faCogs\ 技术能力}
% increase linespacing [parsep=0.5ex]
\begin{itemize}[parsep=0.2ex]
  \item \textbf{语言工具}: Java/Python/Go/SQL/Shell/Linux/ClickHouse/Nebula/Git/Redis/Kafka/Flink/K8s/Docker
  \item \textbf{细分领域}: 大数据/云计算/数据挖掘/ELT/数据存储/数据分析/推荐系统/推荐工程
\end{itemize}

% \end{itemize}

\section{\faSuitcase\ 工作经历}
\datedsubsection{\textbf{新浪微博 | \faWeibo\ }  大数据开发工程师}{2021.06-至今}
\begin{itemize}
  \item 热门微博推荐系统日常样本拼接、特征生产、模型训练、排序引擎等服务的开发、维护及优化。
  \item 热门微博推荐数据系统建设,包括Nebula、ES、CK、Kvrocks等集群部署、调优、维护及数据生产。
  \item 热门微博在线学习平台需求开发,包括模型训练流程化、参数服务器、模型上线等功能的开发。
\end{itemize}

\datedsubsection{\textbf{北京市首都高速公路发展集团公司 | \faRoad\ }  系统集成工程师}{2017.09-2018.08}
\begin{itemize}
 % \item \textbf{利用海量用户定位数据,对城市空间及人群移动特征进行研究。}第一个课题是基于香农熵和人群出行模式,构建城市网格与用户矩阵分析城市多样性/流动性分布;可视分析平台前端与可视化基于D3/Vue/Express开发,数据分析与存储采用Python/MySQL/MongoDB技术,为了均衡大数据情况下的页面可视化渲染消耗用canvas替代svg。第二个课题是对海量商场定位数据做人群分类与可视化查询,依据该系统撰写的论文被CIKM 2016(DAVA Workshop)录用,并收录于中科院软件所年会成果集
  \item 参与建设北京市高速公路办公网的升级改造项目,主要负责网络前期调研,网络设计,线路查找,设备安装,网络切割,系统联调等工作,项目按期完成并顺利交付。 
  \item 参与集团公司内部流控系统,态势感知系统,入侵检测系统的搭建部署及系统维护方面的工作,同时为集团的云计算平台和大数据平台提供技术支持和运维工作。
\end{itemize}

%\datedsubsection{\textbf{北京格灵深瞳信息技术有限公司 | DeepGlint},Web开发工程师}{2015.7-2015.9}
%\begin{itemize}
%\item \textbf{独立负责MUSE部门的可视化组件研发。}与平台研发、设计协作完成 DeepGlint Developer 平台可视化图表组件的集成开发,符合完全定制化渲染、响应式布局与实时更新等特点
%\item 利用 D3+Vue+WebGL(Three.js) 尝试实现三维空间的人群移动可视化
%\end{itemize}


%\section{\faUsers\ 实习经历}
%\datedsubsection{\textbf{北京西云数据科技有限公司(AWS宁夏区) | \faAmazon\ }  Premium Support }{2021.03-2021.05}
%\begin{itemize}
%  \item 熟悉使用AWS各类产品(EC2,ELB,S3,VPC,EMR等),调试测试用例,Trouble shooting等。
%  \item 完成部门内部招聘试题的测试工作,完成 AWS 内部各种技能培训,熟悉技术支持工作内容。
%\end{itemize}

%\datedsubsection{\textbf{中国移动-北京分公司 | \faSellsy\ } 大数据分析工程师 }{2019.07-2019.09}
% \role{Golang, Linux}{个人项目,和富帅糕合作开发}
%\begin{onehalfspacing} 北京移动下属的分公司市场部实习,担任大数据开发与分析工作,主要有两个阶段的实习任务:
%\begin{itemize}
% \item 数据分析系统(千人千面用户画像系统)的逻辑测试,构建各种测试用例实现逻辑测试,针对不同数据维度的需求,实现部分用户数据信息的清洗和筛选。 
% \item 根据用户信息预测用户离网情况,利用随机森林算法实现,根据用户消费行为对用户的通信和上网偏好作出分类。
% \end{itemize}
%\end{onehalfspacing}

%\datedsubsection{\textbf{\LaTeX\ 简历模板}}{2015 年5月 -- 至今}
%\role{\LaTeX, Python}{个人项目}
%\begin{onehalfspacing}
%优雅的 \LaTeX\ 简历模板, https://github.com/billryan/resume
%\begin{itemize}
% \item 容易定制和扩展
%  \item 完善的 Unicode 字体支持,使用 \XeLaTeX\ 编译
%  \item 支持 FontAwesome 4.5.0
%\end{itemize}
%\end{onehalfspacing}

\section{\faPencilSquare\ 项目经历}

\datedsubsection{\textbf{ 热门微博样本拼接优化项目 | \faWeibo\ }  Java系统开发工程师 }{2021.10-2022.04}
\begin{onehalfspacing}  基于成本优化,版本迭代,降低训练消费压力等目的,保证数据一致性前提下,优化样本任务。
\begin{itemize}
  \item 数据结构及格式。字符串样本改成PB序列化压缩样本,节省Kafka带宽50\%+,节省机器资源10\%+。
  \item 特征外置存储。特征数据量较大,外置存储到KVRocks,缓解Flink内存压力,节省机器资源15\%+。
  \item 数据压缩。特征外置聚合,ZStd压缩率达40\%;输出聚合,提高Kafka压缩率,降低Kafka带宽。
  \item 提高稳定性。热点事件大流量会导致Flink反压,数据堆积,解偶数据处理逻辑,拆分数据清洗过程。
  \item 提高可用性。配置项动态配置,推荐业务覆盖范围广,机房跨度大,在大型热点事件时,需动态降级。
 \end{itemize}
\end{onehalfspacing}
\begin{onehalfspacing}  目标超额达成,降低训练消费压力;提高样本任务的稳定性和可用性;深入了解了KVRocks,Flink广播流、ProtoBuffer等技术。
\end{onehalfspacing}

\datedsubsection{\textbf{ 热门微博流数据分析平台 | \faWeibo\ }  Java、Go 系统开发工程师 }{2023.01-至今}
\begin{onehalfspacing}  负责推荐系统数仓DWS层建设,存储热门样本、关注样本、视频样本、曝光、用户画像、物料标签、用户行为等数据。用于数据分析,报表生产等,熟悉CK常用操作函数,各种表引擎,了解CK底层原理。
\begin{itemize}
 \item CK部署运维:基于K8s部署CK集群,基于Prometheus+Grafana部署监控报警,基于MetaBase构建数据可视化和用户权限管理系统。
 \item 数据生产:针对不同数据格式数据流,解析、抽取、聚合后实时写入CK集群,建立分布式表并使用Flink写入数据以解决单机写入瓶颈。
 \item SQL优化:针对算法同学数据分析需求编写SQL语句实现,并对日常Sql进行优化,基于共有计算逻辑构建物化视图等。
\end{itemize}
\end{onehalfspacing}

\datedsubsection{\textbf{ 交互式推荐及用户回看项目 | \faWeibo\ }  Java系统开发工程师 }{2022.03-2023.03}
\begin{onehalfspacing}  负责从无到有搭建ES集群,通过Flink写入搜索、曝光、全站物料数据到ES,定义封装查询接口,后期负责集群、查询耗时优化等。具体内容如下:
\begin{itemize}
 \item 数据迭代问题,引入ILM,周期管理数据。
 \item 集群优化,禁用内存交换,调优JVM参数,避免堆外内存过小降低Lucene检索性能。
 \item 索引模版,意图识别解析出用户请求内容,根据一二三级标签以及搜索关键词,查询索引模版。
 \item 索引预热系统,通过构造轻量化的查询请求,优化低QPS查询导致的单次查询耗时高问题。
 \item 过已读功能,请求batch中已读量未知,需多次请求,使用Scroll深分页缓存上次查询结果。
\end{itemize}
\end{onehalfspacing}
\begin{onehalfspacing}  项目已顺利上线,见首页推荐页“点击一下,「小微」带你进入推荐新体验”;产品层面的各类需求顺利上线,优化集群耗时,深入了解了ES的一些优化细节。
\end{onehalfspacing}

\datedsubsection{\textbf{ 热门推荐样本实时性改进项目 | \faWeibo\ }  Java系统开发工程师 }{2022.02-2022.11}
\begin{onehalfspacing}  负责热门推荐样本加速,40min的固定窗口拼接样本,改造成更实时样本,输出专利一篇。流程如下:
\begin{itemize}
  \item 更改样本拼接时间,以下发曝光为起始拼接30min输出。
  \item 更改样本拼接为以真实阅读为起始,改固定拼接窗口10min,改样本拼接方式。
  \item 更改样本以Request形式输出,按用户请求输出,下一刷来即输出,实时性更高。
\end{itemize}
\end{onehalfspacing}
\begin{onehalfspacing} 将样本改为以真实阅读形式,各项指标提升:点击3\%左右,互动2\%左右,时长3\%左右,节省样本拼接资源10\%左右。
\end{onehalfspacing}

\datedsubsection{\textbf{ 图存储项目 | \faWeibo\ }  Java系统开发工程师 }{2021.10-2022.05}
\begin{onehalfspacing}  负责集群搭建查询优化,写入用户行为数据,数据再生产,输出专利一篇。具体内容如下:
\begin{itemize}
 \item 部署运维:图存储调研、选型、架构设计,基于K8s搭建Nebula集群,监控报警,可视化查询平台。
 \item 数据生产:基于Flink构建数据解析、抽取流程,利用Flink窗口等生产用户行为数据,后期负责特征及序列生产,兴趣试探特征生产等。
\end{itemize}
\end{onehalfspacing}
\begin{onehalfspacing} 作为基础数据应用到推荐各环节,召回、粗排、精排和推荐策略都在广泛应用图数据,实现行为数据的查询及特征的生成。
\end{onehalfspacing}


\datedsubsection{\textbf{ 热门微博在线学习平台 | \faWeibo\ }  Java系统开发工程师 }{2023.01-至今}
\begin{onehalfspacing}  负责推荐流在线学习平台前后端开发,应用SpringBoot、Vue、Redis等技术,服务算法同学日常提交CPU训练任务,GPU训练任务,还提供模型上线、参数服务器日常操作以及其他常用工具等。
\begin{itemize}
 \item 分布式训练:将原来脚本化提交训练任务流程化,通过责任链的设计模式实现提交训练任务,并根据组内不同业务线设计不同训练模版,提高系统的可拓展性。
 \item GPU训练:GPU训练子项目通过前端页面提交训练参数,后端项目管理GPU训练任务的状态,通过与K8s交互来实现对Pod内任务状态的管理。
 \item 日常工具:Kafka Topic及Groupid管理功能,机器利用率查询,特征Hash计算功能,模型上线管理,参数服务器管理等。
\end{itemize}
\end{onehalfspacing}

\datedsubsection{\textbf{ 热门微博特征生产项目 | \faWeibo\ }  Java系统开发工程师 }{2024.01-至今}
\begin{onehalfspacing}  负责推荐流、关注流、视频流日常算法需要的实时、离线特征生产工作,主要负责与产品对接需求,评估需求,并与平台方协作开发特征。
\begin{itemize}
 \item 离线特征:使用SQL的离线表聚合,自定义UDF函数,SQL语句实现等工作,主要生产用户、微博的计数特征、画像特征、行为序列特征等,期间生产离线特征有 50+。
 \item 实时特征:使用Flink基于在线Kafka实时数据流,在窗口内聚合、统计、去重等功能,主要生产用户的实时点击、互动、阅读等行为特征,期间生产在线特征有10+。
 \end{itemize}
\end{onehalfspacing}


% \begin{onehalfspacing}
% \end{onehalfspacing}

% \datedsubsection{\textbf{DID-ACTE} 荷兰莱顿}{2015年3月 - 2015年6月}
% \role{本科毕业设计}{LIACS 交换生}
% 利用结巴分词对中国古文进行分词与词性标注,用已有领域知识训练形成 classifier 并对结果进行调优
% \begin{onehalfspacing}
% \begin{itemize}
%   \item 利用结巴分词对中国古文进行分词与词性标注
%   \item 利用已有领域知识训练形成 classifier, 并用分词结果进行测试反馈
%   \item 尝试不同规则,对 classifier 进行调优
% \end{itemize}
% \end{onehalfspacing}

\section{\faTrophy\ 获奖/证书}
% increase linespacing [parsep=0.5ex]
%\begin{itemize}[parsep=0.2ex]
%   \item LeetCodeOJ Solutions, \textit{https://github.com/hijiangtao/LeetCodeOJ}
%  \item 第三届中国软件杯大学生软件设计大赛\textbf{全国一等奖}( \textit{http://www.cnsoftbei.com/} ),2014 年8月
%  \item 中国机器人大赛创意设计大赛\textbf{全国特等奖}( \textit{http://www.rcccaa.org/} ),2013年8月
%   \item 中国机器人大赛暨Robocup公开赛(武术擂台赛)全国一等奖,2013年10月
 % \item 第11届北京理工大学“世纪杯”竞赛学生课外科技作品竞赛\textbf{特等奖},2013年8月
  %\item VIS Components for security system, \textit{https://hijiangtao.github.io/ss-vis-component/}
  %\item 个人博客:\textit{https://hijiangtao.github.io/},更多作品见 \textit{https://github.com/hijiangtao}
   % \end{itemize}
%\begin{onehalfspacing}  获奖如下:

\begin{itemize}[parsep=0.2ex]
  \item 2022.07 新浪微博机器学习研发部2022年度成长发展奖
  \item 2019.12-2020.12华为云2019和2020年度4星级体验官
  \item  2019.12 北京工业大学优秀研究生奖
  \item 2019.12 北京工业大学研究生学习优秀奖(一等)
  \item 2017.06 河北农业大学校级优秀毕业论文(设计)
  \item 2016.11 河北农业大学校级二等奖学金	
 % \end{itemize}
%  \end{onehalfspacing}
%\begin{onehalfspacing} 证书如下:
%\begin{itemize}[parsep=0.2ex]
  %\item 2021.11 CKA(Certified Kubernetes Administrator)
  \end{itemize}
% \end{onehalfspacing}
%   \item 电视节目"爸爸去哪儿"可视化分析展示, \textit{https://hijiangtao.github.io/variety-show-hot-spot-vis/}


% \section{\faHeartO\ 项目/作品摘要}
% \section{项目/作品摘要}
% \datedline{\textit{An Integrated Version of Security Monitor Vis System}, https://hijiangtao.github.io/ss-vis-component/ }{}
% \datedline{\textit{Dark-Tech}, https://github.com/hijiangtao/dark-tech/ }{}
% \datedline{\textit{融合社交网络数据挖掘的电视节目可视化分析系统}, https://hijiangtao.github.io/variety-show-hot-spot-vis/}{}
% \datedline{\textit{LeetCodeOJ Solutions}, https://github.com/hijiangtao/LeetCodeOJ}{}
% \datedline{\textit{Info-Vis}, https://github.com/ISCAS-VIS/infovis-ucas}{}


% \section{\faInfo\ 社会实践/其他}
%\section{\faInfo\ 社区参与/实践其他}
% increase linespacing [parsep=0.5ex]
%
%% Reference
%\newpage
%\bibliographystyle{IEEETran}
%\bibliography{mycite}
\end{document}
